\title{Monty Hall is not Hard}
\author{A.C.}
\date{}

\documentclass[12pt]{article}

\begin{document}
\maketitle


\section{Traditional Framing}
The Monty Hall problem is a classic of introductory probability theory, traditionally used
to teach students that nothing makes sense and math is bullshit. Competent teachers may twist
the message somewhat to teach the more useful lesson that math is frequently subtle, and it
is important to be careful lest we arrive confidently at wrong answers.

In either case, the problem is as follows. You are a contestant on a gameshow, looking at 
three closed doors. You are told that behind two of the doors there is a goat, and behind one
of the doors there is a ferrari. At the end of the game you will have opened one of the doors,
and be allowed to keep what you found behind it. You start the game by picking a door. The host
will then open one of the other doors to reveal a goat, and ask you if you would like to open 
the door you started on (\textbf{stay}) or if you would like to open the other as yet unopened door 
(\textbf{switch}). Assuming you prefer sports cars to goats, should you switch or stay? What are the 
probabilities of walking away with the ferrari for each decision?

For people unfamiliar with the problem the most common instinct is to say that it doesn't matter,
both doors are the same, so the probability of scoring a ferrari is $\frac{1}{2}$ in either case.
People familiar with the problem know that it is better to switch, which gives a $\frac{2}{3}$ win
rate, rather than to stay with a $\frac{1}{3}$ win rate. They know this, because somebody took the
time to enumerate all the possible outcomes for them and count which would have \textbf{switch} winning and 
which would have \textbf{stay} winning, and the ratios were $\frac{2}{3}$ and $\frac{1}{3}$. Such an 
explanation is correct, and also does nothing to explain why ``they're the same'' was wrong. And
if your brain is anything like ours, it will continue scream ``but they're the same'' for quite
some time after you've learned the solution.

\subsection{Counting Possibilities}

We sketch a counting based solution here, but only for completeness. Readers are encouraged to skip 
to the next section for a more human-friendly, but still rigorous, solution.

A game of monty hall can be described by the door the contestant first selected, and by which door
the car happened to start behind. We chart below the outcomes for every possible game of Monty Hall.
\begin{center}
\begin{tabular}{l|l|l}
Contestant Door & Car Door & Winning Strategy \\\hline
1 & 1 & \textbf{stay}\\
1 & 2 & \textbf{switch}\\
1 & 3 & \textbf{switch}\\\hline
2 & 1 & \textbf{switch}\\
2 & 2 & \textbf{stay}\\
2 & 3 & \textbf{switch}\\\hline
3 & 1 & \textbf{switch}\\
3 & 2 & \textbf{switch}\\
3 & 3 & \textbf{stay}\\
\end{tabular}
\end{center}
Noting that the car is equally likely to be placed in position 1, 2, or 3, we see by rote calculation
that \textbf{switch} wins $\frac{2}{3}$ games while \textbf{stay} wins the remaining $\frac{1}{3}$.

\section{Intuitive Solution}
The Monty Hall problem is a parlor trick designed to make people give the wrong answer, and like
most parlor tricks it works by hiding important details of what's happening from its audience. More 
specifically, the reason people want \textbf{switch} to equal \textbf{stay} is that the human brain, and especially
the human brain trained in mathematics or physics, \emph{loves} symmetry. And the Monty Hall game
ends with a powerfully symmetric image -- two closed doors with a goat behind one and a ferrari behind 
the other. All the asymmetry derives from one step in the middle, the one where the host opens a goat
door, and is remarkably easy to overlook.

Since the problem statement is pathologically bad, all we have to do is restate it equivalently and
then solve the new problem. We propose another gameshow, named Monty Cargo Crates. The game starts
with you standing in front of three cargo crates. As before, two of them contain goats, and one contains
a ferrari. Like before, you start the game by picking one of the crates. Once you've selected a crate,
the host signals some longshoremen to bundle the two remaining crates together into a big crate. Then
he runs into the big crate and frees a goat from inside it. Last, you are given the choice between
switching to the big crate and the little crate.

We've made progress; the addition of a physical partition between the selected and un-selected sets to
match the informational partition is quieting the voice in our head demanding we solve by symmetry, but
it's not solvable by inspection yet, so we continue.

Suppose now that Monty Cargo Crates has been running for some time, and that on one of the shows the
host broke his leg in a literal sense. Now running into the big crate to free a goat is troublesome,
but he realizes that he can offload the work on the contestant. He decides to skip the free-a-goat
step and just tell contestants that pick the big bundle that if there are two goats inside to please
only take one. The game is now as follows: A contestant selects a crate, the two other crates are 
bundled together, and the contestant can either choose to search the bundle for a ferrari or her
single selected crate. It is obvious at this point that the big bundle will have the ferrari $\frac{2}{3}$ of
the time and the lone crate will have it only $\frac{1}{3}$ of the time. Since the big-bundle in Monty
Cargo Crates maps onto \textbf{switch} in Monty Hall, we expect the same solution there.

\end{document}
